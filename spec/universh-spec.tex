\documentclass[llpt]{article}
\title{\textbf{UNIVERSH Package Specification}}
\author{Aun-Ali Zaidi\\
        Russell Greene}
\date{}

\renewcommand*\contentsname{Table of Contents}

\begin{document}

\maketitle

\newpage

\tableofcontents

\newpage

\section {Preface}

The UNIVERSH Package Format is based on the Debian .deb package format, with various extensions. Debian's .deb format was used for its
relatively large user base and robust packaging utilities, perfect for our use case. Additionally, converting Debian packages becomes
trivial with such a simmilar format.
\newline
\newline
UNIVERSH is designed to be a portable, efficient distribution system with the prime focus on providing platform-agnostic
binaries. It achieves this by compiling all binaries into a specialized version of LLVM IR, which in turn is then converted to machine code during
installation. For more info on the modified LLVM IR, you may read through the PortIR spec.

\section {Package File Format}

TODO: Add content here and subsections.

\section {Package Installation Triggers}

TODO: Add content here and subsections.

\section {Package Management Utilities}

TODO: Add content here and subsections.

\section {Package Encryption and Signing}

TODO: Add content here and subsections.

\section {Package Distribution Infrastructure}

TODO: Add content here and subsections.

\end{document}
